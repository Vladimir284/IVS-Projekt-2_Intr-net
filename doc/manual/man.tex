\documentclass[11pt, a4paper]{article}

\usepackage[utf8]{inputenc}
\usepackage[top=2.5cm, left=1.5cm, text={18cm, 25cm}]{geometry}
\usepackage[IL2]{fontenc}
\usepackage{amsmath}
\usepackage{amsthm}
\usepackage{caption}
\usepackage[unicode]{hyperref}
\usepackage{graphicx}

\renewcommand{\figurename}{Button Image}
\title{proj2}
\author{Intr-net}
\date{March 2021}
\begin{document}
    \begin{titlepage}
        \begin{center}
            \vspace*{1cm}

            \huge
            \Huge F\huge AKULTA INFORMAČNÍCH TECHNOLOIÍ\\
            \hspace{0.2cm}
            \Huge V\huge YSOKÉ UČENÍ TECHNIOCKÉ V \Huge B\huge RNĚ\\
            \vspace{\stretch{0.382}}
            CALCULATOR \\
            MANUAL
            \vspace{\stretch{0.618}}


        \end{center}
        {\LARGE 2021 \hfill
        Intr-net}
    \end{titlepage}

    \newpage

    \tableofcontents


    \section{Install}
    \label{sec:install}
    In this section we will take a closer look how to install Calculator 0.1 to your computer


    \section{Usage}
    \t Usage of buttons with mathematical functions.
    For more detailed description, open refman.pdf.
    \label{sec:usage}

    \subsection{Math operations}
    \label{subsec:mathOperations}
        First you need to enter number then some math operation. If the operation is unary (factorial, absolute value) the result will be automatically displayed. In case of binary operation, after operation you have to enter another number and then click on equal button.
    \subsubsection{Plus}
    \label{subsubsec:plus}
        \begin{figure}[h]
        \caption{Plus}
        \includegraphics[scale = 0.2]{plus.gif}
        \centering
        \label{fig:plus}
    \end{figure}
        This button will add two numbers.
    \subsubsection{Minus}
    \label{subsubsec:minus}
        \begin{figure}[h]
            \caption{Minus}
            \includegraphics[scale = 0.2]{minus.gif}
            \centering
            \label{fig:minus}
        \end{figure}
        This button will subtract two numbers.
\newpage
    \subsubsection{Times}
    \label{subsubsec:times}
        \begin{figure}[h]
            \caption{Multiply}
            \includegraphics[scale = 0.2]{multiply.gif}
            \centering
            \label{fig:mul}
        \end{figure}
        This button will multiply two numbers
    \subsubsection{Divide}
    \label{subsubsec:division}
        \begin{figure}[h]
            \caption{divide}
            \includegraphics[scale = 0.2]{divide.gif}
            \centering
            \label{fig:div}
        \end{figure}
        This button will divide two number.
    \subsubsection{Factorial}
    \label{subsubsec:factorial}
        \begin{figure}[h]
            \caption{Factorial}
            \includegraphics[scale = 0.2]{factorial.gif}
            \centering
            \label{fig:fact}
        \end{figure}
        This button will calculate factorial of a number.
    \subsubsection{Absolute value}
    \label{subsubsec:absolute value}
        \begin{figure}[h]
            \caption{Absolute value}
            \includegraphics[scale = 0.2]{abs.gif}
            \centering
            \label{fig:abs_v}
        \end{figure}
        This button will return absolute value of a number.
    \subsubsection{Power}
    \label{subsubsec:power}
        \begin{figure}[h]
            \caption{Power}
            \includegraphics[scale = 0.2]{n-th_power.gif}
            \centering
            \label{fig:pow}
        \end{figure}
        This button will do power. First number is an exponent and second number is base.
    \subsubsection{Nth root}
    \label{subsubsec:nthroot}
        \begin{figure}[h]
            \caption{Nth root}
            \includegraphics[scale = 0.2]{n-th_root.gif}
            \centering
            \label{fig:root}
        \end{figure}
        This button will do nth root. First number is root and second number is base.
    \newpage

    \subsection{Memory oprations}
    \t Usage of buttons with memory operations.
    For more detailed description, open refman.pdf.
    \label{subsec:memoryoperations}

    \subsubsection{Clear Display }

    \label{subsubsec:cleardisplay}

    \begin{figure}[h]
        \caption{Clear display}
        \includegraphics[scale = 0.2]{clear_display}
        \centering
        \label{fig:ac}
    \end{figure}
    This button will clear the display. Last saved answer will remain unchanged

    \subsubsection{Clear Memory}
    \label{subsubsec:clearmemory}

    \begin{figure}[h]
        \caption{Clear memory}
        \includegraphics[scale = 0.2]{clear_memory}
        \centering
        \label{fig:c}
    \end{figure}
    This button will reset last saved answer to zero and will also clear the display

    \subsubsection[scale = 0.2]{Hint}
    \label{subsubsec:hint}


    \begin{figure}[h]
        \caption{Hint button}
        \includegraphics[scale = 0.2]{hint}
        \centering
        \label{fig:h}
    \end{figure}

    This button will open up this manual.

    \newpage

    \subsection{Equal}
    \label{subsec:equal}


    \section{Uninstall}
    \label{sec:uninstall}

\end{document}